In recent years, the progress of \acp{LLM} has been remarkable, especially in tasks that require logical reasoning and structured problem-solving, such as solving mathematical questions and logical puzzles~\cite{panLogicLMEmpoweringLarge2023,zebazeTreeProblemsImproving2024,yuFlowReasoningTraining2025}. These models have also shown a strong capability to engage in human-like question-and-answer dialogues, highlighting their adaptability across various reasoning processes~\cite{haoRecentProgressLeveraging2022, panLogicLMEmpoweringLarge2023,zebazeTreeProblemsImproving2024,yuFlowReasoningTraining2025}.

Human reasoning can be categorized into two main types: vertical and lateral thinking. Vertical thinking, also known as linear or logical thinking, involves a sequential and analytical approach based on rationality, logic, and established rules~\cite{jiangBRAINTEASERLateralThinking2023}. On the other hand, lateral thinking, often referred to as "thinking outside the box," is a creative and divergent process that challenges conventional methods and explores alternative perspectives~\cite{jiangBRAINTEASERLateralThinking2023}.

Despite the significant achievements of \acp{LLM} in tasks involving implicit and complex reasoning, such as mathematical problem-solving, most research has concentrated on vertical thinking, leaving lateral thinking puzzles less explored~\cite{jiangBRAINTEASERLateralThinking2023,chenWeakevalStrongEvaluatingEliciting2024,huangLatEvalInteractiveLLMs2024}. To address this gap, \textcite{jiangBRAINTEASERLateralThinking2023} introduced \citetitle{jiangBRAINTEASERLateralThinking2023}, a multiple-choice question-answering challenge designed to assess a model's ability to demonstrate lateral thinking and overcome default commonsense associations.

\textcite{jiangBRAINTEASERLateralThinking2023} evaluated models like Flan-T5, RoBERTa, and \acs{GPT}~3.5, but did not include open models such as \acs{LLaMA}, \acs{Phi}, and \acs{Qwen}~\cite{jiangBRAINTEASERLateralThinking2023}. However, these open
models have shown superior performance on various benchmarks~\cite{OpenLLMLeaderboard},
prompting an important question about their effectiveness on the \citetitle
{jiangBRAINTEASERLateralThinking2023} task.

This paper investigates the performance of six open \acp{LLM} on the \citetitle{jiangBRAINTEASERLateralThinking2023} task, comparing their results to those of the original models used by \textcite{jiangBRAINTEASERLateralThinking2023}, specifically \acs{GPT}~3.5. The evaluation examines both initial performance and performance after prompt engineering, particularly through \ac{ICL} techniques such as Few-Shot Learning. These \ac{ICL} methods, which involve providing examples within the prompt to guide the model’s responses, enhance accuracy without the need for task-specific modifications \cite{yinDeeperInsightsUpdates2024,brownLanguageModelsAre2020}. By evaluating these models within the \citetitle{jiangBRAINTEASERLateralThinking2023} task, this paper offers valuable insights into their strengths and limitations, furthering our understanding of their capabilities in creative reasoning tasks.